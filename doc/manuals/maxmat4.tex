\documentclass[12pt]{article}
\usepackage[a4paper]{geometry}
\usepackage{url}
\usepackage{alltt}
\usepackage{xspace}
\usepackage{times}
\usepackage{bbm}
\usepackage{verbatim}
\usepackage{ifthen}
\usepackage{optionman}

\newcommand{\MMFour}{\texttt{maxmat4}\xspace}
\newcommand{\GenomeTools}{\textit{GenomeTools}\xspace}
\newcommand{\MUM}[0]{\textit{MUM}\xspace}
\newcommand{\packedindex}{\textit{packedindex}\xspace}
\newcommand{\Match}[3]{(#3,#1,#2)}

\newcommand{\Showrep}[2]{\begin{array}{@{}c@{}}#1\\ {#2} \end{array}}
\newcommand{\Function}[3]{#1:#2\to#3}
\newcommand{\Size}[1]{|#1|}
\newcommand{\Iff}{if and only if\xspace}
\newcommand{\Repfind}[0]{\texttt{\small repfind}\xspace}
\newcommand{\Suffixerator}[0]{\texttt{\small suffixerator}\xspace}
\newcommand{\Nats}{\textrm{I}\!\textrm{N}}
\newcommand{\Substring}[3]{#1[#2..#3]}
\newcommand{\Subchar}[2]{#1[#2]}

\title{\MMFour: a program to find maximal matches\\
       in Large Sequences Manual}

\author{\begin{tabular}{c}
         \textit{Jiabin Huang}\\
         \textit{Stefan Kurtz}\\[1cm]
         Research Group for Genomeinformatics\\
         Center for Bioinformatics\\
         University of Hamburg\\
         Bundesstrasse 43\\
         20146 Hamburg\\
         Germany\\[1cm]
         \url{kurtz@zbh.uni-hamburg.de}\\[1cm]
        \end{tabular}}
\begin{document}
\maketitle

The \MMFour tools of the \GenomeTools package is a program to
find maximal matches of some minimum length between a reference-sequence 
and a query-sequence using FM-index\cite{Ferragina00opportunisticdata}. It also allows for 
the computation of \MUM-candidates as well as \MUM{s}. This document describes
the options of \MMFour and the output format. 
In the following, the notion
\emph{match}  always refers to \emph{maximal matches} if not stated
otherwise. Match only the characters a, c, g, or t, they can be in upper or in lower case.

\section{The Options}

The program \MMFour is called as follows:

\noindent\texttt{gt}  \noindent\texttt{dev}  
$\MMFour~~[\mathit{options}]~~\mathit{referencepckindex}~~
           \mathit{queryfile}_{1}~~\cdots~~\mathit{queryfile}_{n}$
%\noindent\texttt{gt} \Repfind [\textit{options}] \Showoption{ii}
%\Showoptionarg{indexname} [\textit{options}] 
           
referencepckindex is a packed index file, which is constructed from a given set of subject sequences 
using \packedindex tools of the \GenomeTools package.

The options for \MMFour are as follows:

\begin{list}{}{}

\Option{mum}{}{
Compute \MUM{s}, i.e.\ matches that are unique in both sequences.
}

\Option{mumreference}{}{
Compute \MUM-candidates, i.e.\ matches that are unique
in the reference-sequence but not necessarily in the query-sequence.
}

\Option{maxmatch}{}{
Compute all maximal matches, regardless of their uniqueness.
}

\Option{l}{$q$}{
Set the minimum length $q$ of a match to be reported. $q$ must be a 
positive integer. Only matches of length at least $q$ are reported. 
If this option is not used, then the default value for $q$ is 20.
}

\Option{b}{}{
Compute direct matches and reverse complement matches.
}

\Option{r}{}{
Only compute reverse complement matches.
}

\Option{s}{}{
Show the matching substring.
}

\Option{c}{}{
Report the query-position of a reverse complement match
relative to the original query-sequence. Suppose \(x\) is the 
reference-sequence and \(y\) is the query-sequence. By definition, a
reverse complement match \(\Match{i}{j}{l}\) of \(x\) and \(y\) is a
match of \(x\) and \(\overline{y}\), where \(\overline{y}\) is the
reverse complement of \(y\). By default, a match is reported as a triple
\begin{alltt}
i        j        l
\end{alltt}
Position \(j\) is relative to \(\overline{y}\). With this option,
not \(j\) but \(m-j+1\) is reported, where \(m\) is the length of the 
query-sequence. Now note that position \(m-j+1\) in \(y\) corresponds 
to position \(j\) in \(\overline{y}\).
}

\Option{F}{}{
Forces 4 column output format regardless of the number of reference
sequence inputs, i.e. prefixes every output match with its reference
sequence identifier.
}

\Option{L}{}{
Show the length of the query-sequence on the header line.
}

\Option{version}{}{
Displays \GenomeTools version information.
}

\Option{help}{}{
Show the possible options and exit.
}

\end{list}

Options \Showoption{mum}, \Showoption{mumreference} and
\Showoption{maxmatch} cannot be combined. If none of these options are
used, then the program will default to \Showoption{mumreference}.

Option \Showoption{b} and \Showoption{r} exclude each other. If none
of these two options is used, then only direct matches are reported.
Option \Showoption{c} can only be used in combination with
option \Showoption{b} or option \Showoption{r}.

There must be exactly one reference packed index file given and at least one
query-file.
The reference packed index file must be a pck format, which is constructed from a given set of subject sequences 
using \packedindex tools of the \GenomeTools package. The query-files must be in multiple fasta format.
The query-files are processed one after the other. The uniqueness condition
for the \MUM{s} refers to the entire set of reference-sequences but
only to one query-sequence. That is, if a \MUM is reported, the
matching substring is unique in all reference-sequences and in the currently
processed query-sequence. The uniqueness does not necessarily extend to
all query-sequences.

Match only the characters a, c, g, or t, they can be in upper or in lower case.
The characters are processed case insensitive. That is, a 
lower case character is identified with the corresponding upper case 
character. The sequence output via option \Showoption{s} is always
in lower case. 
All characters except for $a$, $c$, $g$, and $t$ are replaced by
a unique character which is different for the reference-sequences and
the query-sequences. This prevents false matches involving, for
example, the wildcard symbols $s$, $w$, $r$, $y$, $m$, $k$, $b$, 
$d$, $h$, $v$, and $n$ often occurring in DNA sequences.

\subsection*{Important Remark} 
The user of the program should carefully choose the least length parameter \(\ell\).
The number of maximal matches exponentially decreases with increasing
\(\ell\).

\section{Output format}
Suppose we have two fasta files \texttt{testdata/at1MB} and
\texttt{testdata/U89959\_genomic.fas}. The first file contains 
a collection of ESTs from Arabidopsis thaliana, while the second is 
a complete BAC-sequence from the same organism. We want to use the first file as our
original file, with which our reference packed index file product and the second as out query-file.

At first, we product the reference packed index file from 
fasta files \texttt{testdata/at1MB} with 
\packedindex tools of the \GenomeTools package\cite{packedindexmanual}.

\begin{verbatim}
gt packedindex mkindex -tis -ssp -des -sds -indexname at1MBpck
-db testdata/at1MB -sprank -dna -pl -bsize 10 -locfreq 32 -dir rev 
\end{verbatim}

Now let us look for direct matches and reverse complement
matches in the reference and in the query sequences.
The length of the matches should be at least 24 (options \Showoption{b} 
and \Showoption{l} 24). 

The corresponding program call is as follows:

\begin{verbatim}
gt dev maxmat4 -b -l 24 at1MBpck testdata/U89959_genomic.fas
\end{verbatim}

The standard output format that results from running \MMFour 
on a single reference sequence with the above option is as follows:

\begin{small}
\begin{verbatim}
> gi|1877523|gb|U89959.1| Arabidopsis thaliana
  gi|3450378|gb|AI100417.1|AI100417       416     71282        26
  gi|3449674|gb|AI099935.1|AI099935       119     72539       215
  gi|3450080|gb|AI100119.1|AI100119       221     72776        31
  gi|3449725|gb|AI099986.1|AI099986       206     77816        38
  gi|3449578|gb|AI099839.1|AI099839        70    105387        42
  gi|3449578|gb|AI099839.1|AI099839       171    105652        32
  gi|3449578|gb|AI099839.1|AI099839       226    105707       145
> gi|1877523|gb|U89959.1| Arabidopsis thaliana Reverse
  gi|3449725|gb|AI099986.1|AI099986       205     29122        37
  gi|3450307|gb|AI100346.1|AI100346       168    101871        24
\end{verbatim}
\end{small}

For each query sequence, the corresponding header is reported on each line 
beginning with a \texttt{\symbol{62}} symbol, even if there are no matches corresponding to this sequence. 
Reverse complemented matches follow a query header that has the keyword Reverse 
following the sequence tag, thus creating two headers for each query sequence and 
alternating forward and reverse match lists. 
If the reference-file is a multiple fasta file with more than one sequence,
then the header line of the reference-sequence is also reported.
For each match, the three columns list the position in the reference sequence, 
the position in the query sequence, and the length of the match respectively. 
Reverse complemented query positions are reported relative to the reverse 
of the query sequence unless the \Showoption{c} option was used. 
As was stated above the \Showoption{L} option adds the sequence lengths to the header line 
and the \Showoption{s} option adds the match strings to the output, if these options 
were used the program call would be as follows:

\begin{verbatim}
gt dev maxmat4 -b -l 24 -L -s -c at1MBpck testdata/U89959_genomic.fas
\end{verbatim}

the format would be as follows:

\begin{small}
\begin{verbatim}
> gi|1877523|gb|U89959.1| Arabidopsis thaliana  Len = 106973                                    
  gi|3450378|gb|AI100417.1|AI100417       416     71282        26
tcatcatcatcatcatcatcatcata
  gi|3449674|gb|AI099935.1|AI099935       119     72539       215
ttaaaactccaaccttggaagattttaggagaatgagagcgacacgctctgtgcttcttttcctt ...
  gi|3450080|gb|AI100119.1|AI100119       221     72776        31
atatatatataaagcaaacatgttccagaat
  gi|3449725|gb|AI099986.1|AI099986       206     77816        38
atatatatatatatatatatatatatatatatatatgt
  gi|3449578|gb|AI099839.1|AI099839        70    105387        42
ctcattggctggggaacttctgatgacggcgaagattattgg
  gi|3449578|gb|AI099839.1|AI099839       171    105652        32
gaggaacgaacgaatgtggcattgaacagagt
  gi|3449578|gb|AI099839.1|AI099839       226    105707       145
gaagaacgtatttaaaggtattaccacttcagatgatcttctggtttcctcagtctaaacaagac ...
> gi|1877523|gb|U89959.1| Arabidopsis thaliana Reverse  Len = 106973
  gi|3449725|gb|AI099986.1|AI099986       205     77852        37
catatatatatatatatatatatatatatatatatat
  gi|3450307|gb|AI100346.1|AI100346       168      5103        24
tctcttcttcttcttcttcttctt
\end{verbatim}
\end{small}

Where the length of each query is noted after the $Len$ keyword and 
the match string is listed on the line after its match coordinates. 
Note that the ellipsis marks are not part of the actual output, 
but added to fit the output into the manual. 
Finally, with the \Showoption{F} option (Force 4 column output format), 
the reference sequence identifier is also reported. 
This is placed at the end of each match line, 
creating an four column command line as follows:

\begin{verbatim}
gt dev maxmat4 -b -l 24 -L -s -c -F 
at1MBpck testdata/U89959_genomic.fas
\end{verbatim}

creating an four column output format as follows:
\begin{small}
\begin{verbatim}
> gi|1877523|gb|U89959.1| Arabidopsis thaliana  Len = 106973                                    
       416     71282        26       561
tcatcatcatcatcatcatcatcata
       119     72539       215      1265
ttaaaactccaaccttggaagattttaggagaatgagagcgacacgctctgtgcttcttttcctt ...
       221     72776        31       859
atatatatataaagcaaacatgttccagaat
       206     77816        38      1214
(output continues ...)
\end{verbatim}
\end{small}


\bibliographystyle{unsrt}
\bibliography{maxmat4}
\end{document}
